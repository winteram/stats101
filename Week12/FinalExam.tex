\documentclass[11pt]{exam}
\usepackage{Sweave}
\addtolength{\textwidth}{-0.5in} 
 
\noprintanswers
\addpoints
\pointformat{(\thepoints)}

\begin{document}


\firstpageheader{Name:\enspace\makebox[2in]{\hrulefill} \\ Statistics
  \\ BT221}{\Large Final Exam } {\ifprintanswers \textbf{Answer Key} \fi
  \\ UNIQUEKEY}
\runningheader{Statistics \\ BT221}{\Large Final Exam }{\ifprintanswers
  \textbf{Answer Key} \fi \\ UNIQUEKEY} 
\headrule 
\cfoot{\thepage}

\begin{questions}


%%%%%%%%  Sample space

    \question[2] A creative musician decides to blindly throw sandbags at a
    piano and record what notes are played.  If you treat this as a random
    variable, what is the sample space?
    \begin{solutionordottedlines}[0.3in]
      All of the notes on a piano.
    \end{solutionordottedlines}


%%%%%%%%  Probabilities



% latex table generated in R 2.13.0 by xtable 1.5-6 package
% Wed Dec 14 14:12:56 2011
\begin{table}[ht]
\begin{center}
\begin{tabular}{|c|ccccc||c|}
  \hline
 & A & B & C & D & F & Sum \\ 
  \hline
Arts & 14 & 22 & 23 & 24 & 12 & 95 \\ 
  Sciences & 43 & 54 & 41 & 9 & 8 & 155 \\ 
   \hline
\hline
Sum & 57 & 76 & 64 & 33 & 20 & 250 \\ 
   \hline
\end{tabular}
\caption{Counts of Arts and Science students with each grade}
\label{tab:focus}
\end{center}
\end{table}

    \question A recent survey of students found their focus in college,
    Arts or Sciences, and their current overall letter grade.  The counts
    are shown in Table~\ref{tab:focus}.
    \begin{parts}
      \part[3] What is the probability that a student has an Arts focus
      and is getting a C?
      \begin{solutionordottedlines}[0.3in] 
[1] "23 / 250  =  0.092"      \end{solutionordottedlines}
      \part[3] What is the (marginal) probability that a student is getting a B?
      \begin{solutionordottedlines}[0.3in] 
[1] "76 / 250  =  0.304"      \end{solutionordottedlines}
      \part[3] What is the (conditional) probability that a student is getting a B
      \emph{given they have a Sciences focus}?
      \begin{solutionordottedlines}[0.3in] 
[1] "54 / 155  =  0.348387096774194"      \end{solutionordottedlines}
      \part[3] What is the (marginal) probability that a student has an
      Arts focus?
      \begin{solutionordottedlines}[0.3in] 
[1] "95 / 250  =  0.38"      \end{solutionordottedlines}
      \part[3] What is the (conditional) probability that a student has an Arts focus
      \emph{given they're getting a B}?
      \begin{solutionordottedlines}[0.3in] 
[1] "22 / 76  =  0.289473684210526"      \end{solutionordottedlines}
    \end{parts}

    \question If students are receiving the same education regardless of
    their focus, then the grade they receive should be independent of their
    focus.  Do a chi-squared test of independence on Table~\ref{tab:focus}.
    \begin{parts}
      \part[5] If focus and grades are independent, what would be the
      expected number of students who had an Arts focus and received a C?
      \begin{solutionordottedlines}[0.3in] 
[1] "95 * 64 / 250  =  24.32"      \end{solutionordottedlines}      
      \part[10] What is the chi-squared statistic for the test of
      independence?
      \begin{solutionordottedlines}[0.3in] 
28.12898      \end{solutionordottedlines}      
      \part[2] What are the degrees of freedom?
      \begin{solutionordottedlines}[0.3in] 
        (2 - 1) * (5 - 1) = 4
      \end{solutionordottedlines}      
      \part[2] What is the critical chi-squared for a Type I error rate of 0.05?
      \begin{solutionordottedlines}[0.3in] 
[1] 9.487729      \end{solutionordottedlines}            
      \part[1] Can you reject the null hypothesis that the students' grades
      are independent of their focus?
      \begin{solutionordottedlines}[0.3in] 
X-squared 
 "Reject"       \end{solutionordottedlines}            
    \end{parts}

%%%%%%%%  Two-sample t-test


%
%

  \question An article in the magazine \emph{Cosmopolitan} recently claimed
  that men and women are different in how they remember colors.  A set of
  20 men and 20 women were shown color swatches, and ten minutes later had
  to try to identify the original color.  The hue distance from the
  original (where negative values indicate more blue and positive values
  indicate more red) was measured for each participant.  On average, men
  chose colors with an average hue distance of
-2.333 and standard deviation of 
2.824.  Women chose colors with an average hue distance of 
-0.285 with a standard deviation of 
2.857.  Assume the variance in the populations are equal.
  \begin{parts}
    \part[1] What are the variables in this study?
    \begin{solutionordottedlines}[0.3in]
      Gender and hue distance
    \end{solutionordottedlines}
    \part[1] What type of variables are they?
    \begin{solutionordottedlines}[0.3in]
      Gender is a categorical variable. \\
      Hue distance is a quantitative variable.
    \end{solutionordottedlines}
    \part[1] What is the null hypothesis?
    \begin{solutionordottedlines}[0.3in]
      There is no difference in hue distance for men and women.
    \end{solutionordottedlines}
    \part[1] What is the alternative hypothesis? 
    \begin{solutionordottedlines}[0.3in]
      There is a difference in hue distance for men and women.
    \end{solutionordottedlines}
    \part[1] What are the degrees of freedom?
    \begin{solutionordottedlines}[0.3in]
      38
    \end{solutionordottedlines}
    \part[2] What is the critical t-value for a two-tailed test with Type I error
    rate ($\alpha$) = 0.05?
    \begin{solutionordottedlines}[0.3in]
[1] 2.024394    \end{solutionordottedlines}
    \part[2] What is the critical t-value for a two-tailed test with Type I error
    rate ($\alpha$) = 0.01?
    \begin{solutionordottedlines}[0.3in]
[1] 2.711558    \end{solutionordottedlines}
    \part[6] What is the t-value for the difference in means?
    \begin{solutionordottedlines}[0.3in]
        t 
-2.279956     \end{solutionordottedlines}
    \part[1] Would you reject the null hypothesis with a two-tailed Type I error
    rate ($\alpha$) = 0.05?
    \begin{solutionordottedlines}[0.3in]
[1] Yes    \end{solutionordottedlines}
    \part[1] Would you reject the null hypothesis with a two-tailed Type I error
    rate ($\alpha$) = 0.01?
    \begin{solutionordottedlines}[0.3in]
[1] No    \end{solutionordottedlines}
  \end{parts}


%%%%%%%%  ANOVA

%

\begin{table}[ht]
\begin{center}
\begin{tabular}{|llcc|}
  \hline
  & Factory & Mean & Std. Dev. \\ 
  \hline
  & Chicago &
97.76&
14.91\\
& Miami &
106.32&
16.84\\
& Dallas &
93.04&
22.06\\
& Topeka &
101.73&
8.65\\
& Portland &
100.64&
5.23\\
\hline
\end{tabular}
\caption{Average and standard deviation of daily production}
\label{tab:widget}
\end{center}
\end{table}


\question A manager of a company with factories located around the U.S. is
interested in whether the factories have different levels of outoput.  She
chooses thirty days out of the year, and monitors the output of each of the
companies. The factory, average output per day, and standard deviation of
the output over the 30 days is shown in Table~\ref{tab:widget}.

\begin{parts}
  \part[6] What is the grand mean return for all investment types in the sample?
  \begin{solutionordottedlines}[0.3in]
[1] 99.898  \end{solutionordottedlines}
  \part[10] What is the between-groups sum of squared deviations?
  \begin{solutionordottedlines}[0.3in]
[1] 725.6406  \end{solutionordottedlines}
  \part[10] What is the within-groups sum of squared deviations?
  \begin{solutionordottedlines}[0.3in]
[1] 218.9425  \end{solutionordottedlines}
  \part[3] What is the F-statistic for this analysis of variance (ANOVA)?
  \begin{solutionordottedlines}[0.3in]
[1] 3.314297  \end{solutionordottedlines}
  \part[2] What are the degrees of freedom (numerator, denominator) for
  this ANOVA?
  \begin{solutionordottedlines}[0.3in]
4,  145  \end{solutionordottedlines}
  \part[2] What is the critical F-vaue for those degrees of freedom?
  \begin{solutionordottedlines}[0.3in]
[1] 2.434065  \end{solutionordottedlines}
  \part[1] Can you reject the null hypothesis?
  \begin{solutionordottedlines}[0.3in]
[1] "Reject"  \end{solutionordottedlines}
\end{parts}

%%%%%%%%  Correlation & Regression

%
 
  \question It has been widely claimed that the more churches there are in
  a neighborhood, the more bars there are.  To test this hypothesis, you
  visit 12 randomly chosen counties in the U.S., and count the number of
  bars and churches in that county (identified by a random ID).  Across all
  counties, there were an average of 
28.58 bars, with a standard deviation of 
4.379.  The actual counts are shown in Table~\ref{tab:church}.


\ifprintanswers
% latex table generated in R 2.13.0 by xtable 1.5-6 package
% Wed Dec 14 14:12:56 2011
\begin{table}[ht]
\begin{center}
\begin{tabular}{|c|c|c|c|c|c|}
  \hline
CountyID & Bars & Churches & B.resid & C.resid & product \\ 
  \hline
1297 & 27.00 & 20.00 & -1.58 & 1.08 & -1.71 \\ 
   \hline
1567 & 34.00 & 24.00 & 5.42 & 5.08 & 27.53 \\ 
   \hline
2359 & 27.00 & 15.00 & -1.58 & -3.92 & 6.19 \\ 
   \hline
4249 & 34.00 & 22.00 & 5.42 & 3.08 & 16.69 \\ 
   \hline
4970 & 23.00 & 15.00 & -5.58 & -3.92 & 21.87 \\ 
   \hline
5300 & 32.00 & 20.00 & 3.42 & 1.08 & 3.69 \\ 
   \hline
5742 & 29.00 & 16.00 & 0.42 & -2.92 & -1.23 \\ 
   \hline
6963 & 20.00 & 17.00 & -8.58 & -1.92 & 16.47 \\ 
   \hline
8523 & 25.00 & 14.00 & -3.58 & -4.92 & 17.61 \\ 
   \hline
9512 & 32.00 & 24.00 & 3.42 & 5.08 & 17.37 \\ 
   \hline
9553 & 31.00 & 21.00 & 2.42 & 2.08 & 5.03 \\ 
   \hline
9812 & 29.00 & 19.00 & 0.42 & 0.08 & 0.03 \\ 
   \hline
\end{tabular}
\caption{Number of bars and churches in random sample of U.S. counties}
\label{tab:church}
\end{center}
\end{table}\else
% 
% latex table generated in R 2.13.0 by xtable 1.5-6 package
% Wed Dec 14 14:12:56 2011
\begin{table}[ht]
\begin{center}
\begin{tabular}{|c|c|c|c|c|c|}
  \hline
CountyID & Bars & Churches & B.resid & C.resid & product \\ 
  \hline
1297 & 27.00 & 20.00 & -1.58 & 1.08 & -1.71 \\ 
   \hline
1567 & 34.00 & 24.00 & 5.42 & 5.08 & 27.53 \\ 
   \hline
2359 & 27.00 & 15.00 & -1.58 & -3.92 & 6.19 \\ 
   \hline
4249 & 34.00 & 22.00 & 5.42 & 3.08 & 16.69 \\ 
   \hline
4970 & 23.00 & 15.00 & -5.58 & -3.92 & 21.87 \\ 
   \hline
5300 & 32.00 & 20.00 & 3.42 & 1.08 & 3.69 \\ 
   \hline
5742 & 29.00 & 16.00 & 0.42 & -2.92 & -1.23 \\ 
   \hline
6963 & 20.00 & 17.00 & -8.58 & -1.92 & 16.47 \\ 
   \hline
8523 & 25.00 & 14.00 &  &  &  \\ 
   \hline
9512 & 32.00 & 24.00 &  &  &  \\ 
   \hline
9553 & 31.00 & 21.00 &  &  &  \\ 
   \hline
9812 & 29.00 & 19.00 &  &  &  \\ 
   \hline
\end{tabular}
\caption{Number of bars and churches in random sample of U.S. counties}
\label{tab:church}
\end{center}
\end{table}\fi

    \begin{parts}
      \part[3] What is the average number of churches in the sample?
      \begin{solutionordottedlines}[0.3in]
18.92      \end{solutionordottedlines}
      \part[10] What is the standard deviation of the churches in the
      sample?
      \begin{solutionordottedlines}[0.3in]
3.502      \end{solutionordottedlines}
      \part[8] Table~\ref{tab:church} shows the beginnings of the
      calculation of covariance.  Finish the table.
      \part[4] What is the covariance between bars and churches?
      \begin{solutionordottedlines}[0.3in]
11.77636      \end{solutionordottedlines}
      \part[4] What is the correlation coefficient between the number of
      churches and bars?
      \begin{solutionordottedlines}[0.3in]
0.7679273      \end{solutionordottedlines}
    \part[6] Calculate the t-value for the correlation.
      \begin{solutionordottedlines}[0.3in]
4.4858      \end{solutionordottedlines}
      \part[1] How many degrees of freedom are there?
      \begin{solutionordottedlines}[0.3in]
        14
      \end{solutionordottedlines}
      \part[2] What is the critical t-value for a two-tailed Type I error rate of 0.05?
      \begin{solutionordottedlines}[0.3in]
2.144787      \end{solutionordottedlines}
      \part[1] Should you reject the null hypothesis that there is no
      relationship between the variables with a two-tailed Type I error
      rate of 0.05?
      \begin{solutionordottedlines}[0.3in]
[1] "Reject"      \end{solutionordottedlines}
      \part[1] Based on this result, would you recommend reducing the
      number of churches in a neighborhood in order to reduce the number of
      bars in the neighborhood?
      \begin{solutionordottedlines}[0.3in]
        No!  Correlation is not causation.
      \end{solutionordottedlines}
    \end{parts}

    \question Given the data shown in Table~\ref{tab:church}, you want to
    predict how many bars or churches there will be in different
    neighborhoods. In Scarborough county, Maine, there are 
21 churches.  In Hancock country, Mississippi, there are 
14 bars.
    \begin{parts}
      \part[3] What is the z-score for the number of churches in
      Scarborough county?
      \begin{solutionordottedlines}[0.3in]
0.5939463      \end{solutionordottedlines}
      \part[3] What is the predicted z-score for the number of bars in
      Scarborough county?
      \begin{solutionordottedlines}[0.3in]
0.4561076      \end{solutionordottedlines}
    \part[4] How many bars would you expect there to be in
      Scarborough county?
      \begin{solutionordottedlines}[0.3in]
30.58      \end{solutionordottedlines}
    \part[10]   How many churches would you expect there to be in Hancock county?
      \begin{solutionordottedlines}[0.3in]
9.97      \end{solutionordottedlines}
    \end{parts}

%%%%%%%%  Autoregression

%


\ifprintanswers
% latex table generated in R 2.13.0 by xtable 1.5-6 package
% Wed Dec 14 14:12:57 2011
\begin{table}[ht]
\begin{center}
\begin{tabular}{|c|c|c|c|c|c|c|c|}
  \hline
Day & Quake.Tweets & q.L1 & q.L2 & Predicted & Residuals & Sq.Regression & Sq.Residual \\ 
  \hline
14 & 61 & 59 & 56 & 62 & -1 & 9 & 1 \\ 
   \hline
13 & 59 & 56 & 51 & 60.87 & -1.87 & 17.057 & 3.497 \\ 
   \hline
12 & 56 & 51 & 59 & 50.82 & 5.18 & 201.072 & 26.832 \\ 
   \hline
11 & 51 & 59 & 60 & 60 & -9 & 25 & 81 \\ 
   \hline
10 & 59 & 60 & 70 & 56.21 & 2.79 & 77.264 & 7.784 \\ 
   \hline
9 & 60 & 70 & 83 & 61.81 & -1.81 & 10.176 & 3.276 \\ 
   \hline
8 & 70 & 83 & 93 & 72.54 & -2.54 & 56.852 & 6.452 \\ 
   \hline
7 & 83 & 93 & 79 & 91.64 & -8.64 & 709.69 & 74.65 \\ 
   \hline
6 & 93 & 79 & 61 & 83.7 & 9.3 & 349.69 & 86.49 \\ 
   \hline
5 & 79 & 61 & 48 & 68.42 & 10.58 & 11.696 & 111.936 \\ 
   \hline
4 & 61 & 48 & 31 & 61.19 & -0.19 & 14.516 & 0.036 \\ 
   \hline
3 & 48 & 31 & 4 & 54.12 & -6.12 & 118.374 & 37.454 \\ 
   \hline
2 & 31 & 4 &  &  &  &  &  \\ 
   \hline
1 & 4 &  &  &  &  &  &  \\ 
   \hline
Mean: & 65 & 62.5 & 57.917 &  &  &  &  \\ 
   \hline
SD: & 13.565 & 16.758 & 23.831 &  &  &  &  \\ 
   \hline
\end{tabular}
\caption{Number of tweets about earthquakes in the days following a major quake}
\label{tab:quake}
\end{center}
\end{table}\else 
% latex table generated in R 2.13.0 by xtable 1.5-6 package
% Wed Dec 14 14:12:57 2011
\begin{table}[ht]
\begin{center}
\begin{tabular}{|c|c|c|c|c|c|c|c|}
  \hline
Day & Quake.Tweets & q.L1 & q.L2 & Predicted & Residuals & Sq.Regression & Sq.Residual \\ 
  \hline
14 & 61 &  &  & 62 & -1 & 9 & 1 \\ 
   \hline
13 & 59 &  &  & 60.87 & -1.87 & 17.057 & 3.497 \\ 
   \hline
12 & 56 &  &  & 50.82 & 5.18 & 201.072 & 26.832 \\ 
   \hline
11 & 51 &  &  & 60 & -9 & 25 & 81 \\ 
   \hline
10 & 59 &  &  & 56.21 & 2.79 & 77.264 & 7.784 \\ 
   \hline
9 & 60 &  &  & 61.81 & -1.81 & 10.176 & 3.276 \\ 
   \hline
8 & 70 &  &  & 72.54 & -2.54 & 56.852 & 6.452 \\ 
   \hline
7 & 83 &  &  & 91.64 & -8.64 & 709.69 & 74.65 \\ 
   \hline
6 & 93 &  &  & 83.7 & 9.3 & 349.69 & 86.49 \\ 
   \hline
5 & 79 &  &  &  &  &  &  \\ 
   \hline
4 & 61 &  &  &  &  &  &  \\ 
   \hline
3 & 48 &  &  &  &  &  &  \\ 
   \hline
2 & 31 &  &  &  &  &  &  \\ 
  1 & 4 &  &  &  &  &  &  \\ 
  Mean: & 65 & 62.5 & 57.917 &  &  &  &  \\ 
  SD: & 13.565 & 16.758 & 23.831 &  &  &  &  \\ 
  \end{tabular}
\caption{Number of tweets about earthquakes in the days following a major quake}
\label{tab:quake}
\end{center}
\end{table}\fi

 
  \question A researcher is interested in modeling how people share
  information about sudden events.  She decides to count the number of
  tweets each day after a major earthquake, as shown in
  Table~\ref{tab:quake}. 
  \begin{parts}
    \part[3] Create two lagged variables based on the time series of tweets.  Keep
    in mind, the most recent day is at the top of the table.
    \part[6] Finish calculating the predicted values.
    \part[6] Finish calculating the residuals.
    \part[6] Finish calculating the squared deviations of the model from the
    average (\emph{Sq.Regression}).
    \part[6] Finish calculating the squared residuals.
    \part[2] What is the sum of squares for the regression?
    \begin{solutionordottedlines}[0.3in]
[1] 1600.387    \end{solutionordottedlines}
    \part[2] What is the sum of squared residuals?
    \begin{solutionordottedlines}[0.3in]
[1] 440.407    \end{solutionordottedlines}
    \part[1] What are the degrees of freedom for the regression?
    \begin{solutionordottedlines}[0.3in]
      2
    \end{solutionordottedlines}
    \part[1] What are the degrees of freedom for the residual?
    \begin{solutionordottedlines}[0.3in]
      12 - 2 - 1 = 9
    \end{solutionordottedlines}
    \part[2] What is the critical F-value for these degrees of freedom?
    \begin{solutionordottedlines}[0.3in]
[1] 4.256495    \end{solutionordottedlines}
    \part[3] What is the mean squared error of the regression?
    \begin{solutionordottedlines}[0.3in]
[1] 800.1935    \end{solutionordottedlines}
    \part[3] What is the mean squared residual error?
    \begin{solutionordottedlines}[0.3in]
[1] 48.93411    \end{solutionordottedlines}
    \part[3] What is the F statistic for this regression?
    \begin{solutionordottedlines}[0.3in]
[1] 16.35247    \end{solutionordottedlines}
    \part[1] Are the number of tweets on one day significantly predicted by the number of
    tweets in the two previous days?
      \begin{solutionordottedlines}[0.3in]
[1] "Significant"      \end{solutionordottedlines}
    \part[5] What is the goodness-of-fit for this autoregressive
      model (i.e., $R^2$)?
    \begin{solutionordottedlines}[0.3in]
[1] 0.7841982    \end{solutionordottedlines}
  \end{parts}

\ifprintanswers
% latex table generated in R 2.13.0 by xtable 1.5-6 package
% Wed Dec 14 14:12:57 2011
\begin{table}[ht]
\begin{center}
\begin{tabular}{rrrrr}
  \hline
 & Estimate & Std. Error & t value & Pr($>$$|$t$|$) \\ 
  \hline
(Intercept) & 18.6089 & 8.5861 & 2.17 & 0.0584 \\ 
  q.L1 & 1.2076 & 0.2342 & 5.16 & 0.0006 \\ 
  q.L2 & -0.5022 & 0.1647 & -3.05 & 0.0138 \\ 
   \hline
\end{tabular}
\caption{Regression Model}
\label{tab:quake-fit}
\end{center}
\end{table}\else
\begin{table}[ht]
\begin{center}
\begin{tabular}{rrrrr}
  \hline
  & Estimate & Std. Error & t value & sig. ? \\ 
  \hline
(Intercept) ($b_0$) &  18.61  &  8.5861  & &\\
Lag 1 ($b_1$) &  1.21  &  0.2342  & &\\
Lag 2 ($b_2$) &  -0.5  &  0.1647  & &\\
  \hline
\end{tabular}
\caption{Regression Model}
\label{tab:quake-fit}
\end{center}
\end{table}
\fi


  \question You also want to know how well each lagged variable predicts the
  current value.
  \begin{parts}
    \part[6] Calculate the t-value for the intercept and both slopes.
    \part[2] What is the critical t-value for these slopes?
    \begin{solutionordottedlines}[0.3in]
[1] 2.262157    \end{solutionordottedlines}
    \part[3] Which of the parameters ($b_0$,$b_1$,$b_2$) are significantly
    different from zero?
    \begin{solutionordottedlines}[0.3in]
[1] "b0 not sig, "[1] "b1 sig, "[1] "b2 sig"    \end{solutionordottedlines}

  \end{parts}

  \question[6] How many tweets about earthquakes would you expect, based on
  the autogregressive model, for day 15?
  \begin{solutionordottedlines}[0.3in]
[1] 62.92  \end{solutionordottedlines}


\end{questions}

\begin{center} \gradetable[v][questions] \end{center}

\end{document}
