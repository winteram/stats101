\documentclass[11pt]{exam}
\usepackage{Sweave}
\addtolength{\textwidth}{-0.3in} 
 
\printanswers
\addpoints
\pointformat{(\thepoints)}

\begin{document}


\firstpageheader{Name:\enspace\makebox[2in]{\hrulefill} \\ Statistics
  \\ BT221}{\Large Exam 1 } {\ifprintanswers \textbf{Answer Key} \fi
  \\ UNIQUEKEY}
\runningheader{Statistics \\ BT221}{\Large Exam 1 }{\ifprintanswers
  \textbf{Answer Key} \fi \\ UNIQUEKEY} 
\headrule 
\cfoot{\thepage}

\begin{questions}

  \question  A pharmaceutical company is mining their own sales data to
  understand the popularity and effectiveness of different drugs being sold
  on the market. The data set records the chemical name of the drug, the
  commerical name of the drug, the number of purchases, the overall rank of
  the total sales of the drug, and the average frequency of illness in the
  drug-taking population compared to the baseline.
  \begin{parts}
    \part[1] Is this a primary or secondary data set?
    \begin{solutionordottedlines}[0.3in]
      Primary
    \end{solutionordottedlines}
    \part[1] What is an elementary unit in this data set?
    \begin{solutionordottedlines}[0.3in]
      A drug
    \end{solutionordottedlines}
    \part[2] How many variables are there in the data set?
    \begin{solutionordottedlines}[0.3in]
      Five
    \end{solutionordottedlines}
    \part[2] What is/are the nominal variables?
    \begin{solutionordottedlines}[0.3in]
      Chemical name, Commercial name
    \end{solutionordottedlines}
    \part[2] What is/are the ordinal variables?
    \begin{solutionordottedlines}[0.3in]
      Rank of sales
    \end{solutionordottedlines}
    \part[2] What is/are the quantitative variables?
    \begin{solutionordottedlines}[0.3in]
      Number of purchases, frequency of illness
    \end{solutionordottedlines}
  \end{parts}

  \question A news agency is investigating government waste, using a data
  set from a watchdog organization that shows the costs in U.S. dollars of
  every government-funded project, as well as the name of the project, the
  project's funding agency (e.g., NSF, DoD, etc.), the classification of
  the project (e.g., military, telecom, etc.), the most recent grade
  received from the funding agency (e.g., A--F), and the number of patents
  arising from the project.
  \begin{parts}
    \part
    \begin{subparts}
      \subpart[1] Is this a primary or secondary data set?
      \begin{solutionordottedlines}[0.3in]
        Secondary
      \end{solutionordottedlines}
      \subpart[1] What is an elementary unit in this data set?
      \begin{solutionordottedlines}[0.3in]
        A government-funded project
      \end{solutionordottedlines}
      \subpart[2] How many variables are there in the data set?
      \begin{solutionordottedlines}[0.3in]
        Six
      \end{solutionordottedlines}
      \subpart[2] What kind of variable is the classification of the project?
      \begin{solutionordottedlines}[0.3in]
        Nominal
      \end{solutionordottedlines}
      \subpart[2] What kind of variable is the grade from the funding agency?
      \begin{solutionordottedlines}[0.3in]
        Ordinal
      \end{solutionordottedlines}
      \subpart[2] What kind of variable is the number of patents generated
      as a result of the project?
      \begin{solutionordottedlines}[0.3in]
        Quantitative
      \end{solutionordottedlines}
    \end{subparts}

    \part
    \begin{subparts}
      \subpart[3] What plot would you use to examine the distribution of
      project costs?
      \begin{solutionordottedlines}[0.3in]
        Box plot or histogram
      \end{solutionordottedlines}
      \subpart[3] What plot would you use to show the average cost for each
      type of project?
      \begin{solutionordottedlines}[0.3in]
        Bar chart
      \end{solutionordottedlines}
      \subpart[3] How would you illustrate the relationship between the
      cost of the project and the number of patents generated?
      \begin{solutionordottedlines}[0.3in]
        Scatterplot
      \end{solutionordottedlines}
      \subpart[2] How would you modify that plot to show the costs and
      patents for different types of projects?
      \begin{solutionordottedlines}[0.3in]
        Use different colors or shapes for the points in the scatterplot
      \end{solutionordottedlines}
      \subpart[3] How would you illustrate the relationship between the
      grade received by the project and its cost?
      \begin{solutionordottedlines}[0.3in]
        Line plot
      \end{solutionordottedlines}
      \subpart[2] How would you modify that plot to show the costs of
      projects in each grade level for different types of projects?
      \begin{solutionordottedlines}[0.3in]
        Use different colors or linetype for the lines in the line plot
      \end{solutionordottedlines}
    \end{subparts}
  \end{parts}

\newpage
%
%

% latex table generated in R 2.13.0 by xtable 1.5-6 package
% Mon Sep 26 00:39:12 2011
\begin{table}[ht]
\begin{center}
\begin{tabular}{rrrllrr}
  \hline
ID & days & age & gender & politics & msg.sent & msg.received \\ 
  \hline
7267 & 94 & 33 & Male & Liberal & 308 & 10 \\ 
  8230 & 627 & 28 & Female & Conservative & 279 & 6 \\ 
  1138 & 39 & 23 & Female & Moderate & 265 & 9 \\ 
  7887 & 376 & 33 & Female & Conservative & 255 & 1 \\ 
  4992 & 705 & 28 & Female & Liberal & 237 & 10 \\ 
  6653 & 363 & 24 & Male & Conservative & 169 & 9 \\ 
  4321 & 337 & 29 & Male & Moderate & 153 & 9 \\ 
  4738 & 925 & 23 & Female & Conservative & 147 & 2 \\ 
  4874 & 502 & 28 & Male & Very liberal & 120 & 7 \\ 
  2528 & 296 & 26 & Male & Very Conservative & 111 & 6 \\ 
  6144 & 789 & 21 & Male & Very Conservative & 91 & 7 \\ 
  8700 & 81 & 26 & Male & Very Conservative & 70 & 1 \\ 
  3739 & 460 & 27 & Female & Liberal & 43 & 9 \\ 
  2654 & 555 & 28 & Male & Very liberal & 12 & 4 \\ 
  2268 & 713 & 19 & Female & Conservative & 11 & 2 \\ 
  4537 & 212 & 26 & Male & Very Conservative & 9 & 6 \\ 
  9302 & 104 & 31 & Female & Moderate & 8 & 4 \\ 
  6195 & 601 & 22 & Female & Moderate & 7 & 5 \\ 
  7466 & 30 & 22 & Female & Moderate & 4 & 7 \\ 
  1795 & 555 & 25 & Female & Moderate & 2 & 9 \\ 
   \hline
\end{tabular}
\caption{Customers of a dating site}
\label{tab:okcupid}
\end{center}
\end{table}%

  Table~\ref{tab:okcupid} shows data collected by a popular dating site on
  its customers for the purpose of improving their service.
  \question
  \begin{parts}
    \part[4] What is the average number of messages sent?
    \begin{solutionordottedlines}[0.3in]
\begin{Schunk}
\begin{Soutput}
[1] 115.05
\end{Soutput}
\end{Schunk}
    \end{solutionordottedlines}
    \part[2] What is the median number of messages sent?
    \begin{solutionordottedlines}[0.3in]
\begin{Schunk}
\begin{Soutput}
[1] 101
\end{Soutput}
\end{Schunk}
    \end{solutionordottedlines}
    \part[2] What is the 75th percentile of the number of messages sent?
    \begin{solutionordottedlines}[0.3in]
\begin{Schunk}
\begin{Soutput}
75% 
169 
\end{Soutput}
\end{Schunk}
    \end{solutionordottedlines}        
    \part[2] What is the 25th percentile of the number of messages sent?
    \begin{solutionordottedlines}[0.3in]
\begin{Schunk}
\begin{Soutput}
25% 
  9 
\end{Soutput}
\end{Schunk}
    \end{solutionordottedlines}
    \part[1] What is the range of the number of messages sent?
    \begin{solutionordottedlines}[0.3in]
\begin{Schunk}
\begin{Soutput}
[1] 306
\end{Soutput}
\end{Schunk}
    \end{solutionordottedlines}
    \part[1] What is the interquartile range of the number of messages sent?
    \begin{solutionordottedlines}[0.3in]
\begin{Schunk}
\begin{Soutput}
75% 
160 
\end{Soutput}
\end{Schunk}
    \end{solutionordottedlines}

\newpage
    \part[10] Draw a boxplot of the number of messages sent, \emph{showing any
    outliers}. Be sure to label key points in the plot, i.e., label the
    points with their values).
    \begin{solutionorbox}[3in]
\begin{Schunk}
\begin{Soutput}
[1] "next highest value: 308"
\end{Soutput}
\end{Schunk}
\includegraphics{Exam1-boxsent}
    \end{solutionorbox}
    \part[8] Draw a histogram of the number of messages \emph{received}
    (not the messages sent!).  There is no need to put the values into bins
    (in other words, the bin size should be 1).
    \begin{solutionorbox}[3in]
\includegraphics{Exam1-histrcvd}
    \end{solutionorbox}
    \part[2] What is the modal number of messages \emph{received}?
    \begin{solutionordottedlines}[0.3in]
\begin{Schunk}
\begin{Soutput}
9 
5 
\end{Soutput}
\end{Schunk}
    \end{solutionordottedlines}
  \end{parts}

\newpage


\ifprintanswers
% latex table generated in R 2.13.0 by xtable 1.5-6 package
% Mon Sep 26 00:39:12 2011
\begin{table}[ht]
\begin{center}
\begin{tabular}{|c|c|c|c|}
  \hline
Opponent & Score & $Score - \mu$ & ${(Score - \mu)}^2$ \\ 
  \hline
Chicago Bulls & 87 & 16 & 256 \\ 
   \hline
Chicago Bulls & 83 & 12 & 144 \\ 
   \hline
Minnesota Timberwolves & 82 & 11 & 121 \\ 
   \hline
Miami Heat & 82 & 11 & 121 \\ 
   \hline
New Jersey Nets & 80 & 9 & 81 \\ 
   \hline
Miami Heat & 79 & 8 & 64 \\ 
   \hline
New Jersey Nets & 75 & 4 & 16 \\ 
   \hline
New Jersey Nets & 74 & 3 & 9 \\ 
   \hline
Minnesota Timberwolves & 74 & 3 & 9 \\ 
   \hline
New Jersey Nets & 73 & 2 & 4 \\ 
   \hline
New Jersey Nets & 70 & -1 & 1 \\ 
   \hline
Chicago Bulls & 68 & -3 & 9 \\ 
   \hline
Chicago Bulls & 67 & -4 & 16 \\ 
   \hline
Minnesota Timberwolves & 66 & -5 & 25 \\ 
   \hline
Minnesota Timberwolves & 65 & -6 & 36 \\ 
   \hline
Chicago Bulls & 64 & -7 & 49 \\ 
   \hline
Minnesota Timberwolves & 63 & -8 & 64 \\ 
   \hline
Miami Heat & 60 & -11 & 121 \\ 
   \hline
Miami Heat & 58 & -13 & 169 \\ 
   \hline
Miami Heat & 50 & -21 & 441 \\ 
   \hline
\hline
 & 71 &  & 1756 \\ 
   \hline
\end{tabular}
\caption{Scores of last 20 Knicks games}
\label{tab:knicks}
\end{center}
\end{table}\else
%
% latex table generated in R 2.13.0 by xtable 1.5-6 package
% Mon Sep 26 00:39:12 2011
\begin{table}[ht]
\begin{center}
\begin{tabular}{|c|c|p{3cm}|p{3cm}|}
  \hline
Opponent & Score &  &  \\ 
  \hline
Chicago Bulls & 87 &  &  \\ 
   \hline
Chicago Bulls & 83 &  &  \\ 
   \hline
Minnesota Timberwolves & 82 &  &  \\ 
   \hline
Miami Heat & 82 &  &  \\ 
   \hline
New Jersey Nets & 80 &  &  \\ 
   \hline
Miami Heat & 79 &  &  \\ 
   \hline
New Jersey Nets & 75 &  &  \\ 
   \hline
New Jersey Nets & 74 &  &  \\ 
   \hline
Minnesota Timberwolves & 74 &  &  \\ 
   \hline
New Jersey Nets & 73 &  &  \\ 
   \hline
New Jersey Nets & 70 &  &  \\ 
   \hline
Chicago Bulls & 68 &  &  \\ 
   \hline
Chicago Bulls & 67 &  &  \\ 
   \hline
Minnesota Timberwolves & 66 &  &  \\ 
   \hline
Minnesota Timberwolves & 65 &  &  \\ 
   \hline
Chicago Bulls & 64 &  &  \\ 
   \hline
Minnesota Timberwolves & 63 &  &  \\ 
   \hline
Miami Heat & 60 &  &  \\ 
   \hline
Miami Heat & 58 &  &  \\ 
   \hline
Miami Heat & 50 &  &  \\ 
   \hline
\hline
 &  &  &  \\ 
   \hline
\end{tabular}
\caption{Scores of last 20 Knicks games}
\label{tab:knicks}
\end{center}
\end{table}\fi

  \question As the newly-hired manager of the New York Knicks basketball
  team, you're interested in how they have been performing recently.
  Table~\ref{tab:knicks} shows the team's scores over the past 20 games.

    \begin{parts}
      \part[4] What is the average score?
        \begin{solutionordottedlines}[0.3in] 
\begin{Schunk}
\begin{Soutput}
[1] 71
\end{Soutput}
\end{Schunk}
        \end{solutionordottedlines}
    \part[1] What is the range of scores?
    \begin{solutionordottedlines}[0.3in]
\begin{Schunk}
\begin{Soutput}
[1] 306
\end{Soutput}
\end{Schunk}
    \end{solutionordottedlines}
      \part[2] What is the median score?
        \begin{solutionordottedlines}[0.3in] 
\begin{Schunk}
\begin{Soutput}
[1] 71.5
\end{Soutput}
\end{Schunk}
        \end{solutionordottedlines}
      \part[2] What is the 75th percentile for score?
        \begin{solutionordottedlines}[0.3in] 
\begin{Schunk}
\begin{Soutput}
75% 
 79 
\end{Soutput}
\end{Schunk}
        \end{solutionordottedlines}
      \part[2] What is the 25th percentile for score?
        \begin{solutionordottedlines}[0.3in] 
\begin{Schunk}
\begin{Soutput}
25% 
 64 
\end{Soutput}
\end{Schunk}
        \end{solutionordottedlines}
    \part[1] What is the interquartile range of scores?
    \begin{solutionordottedlines}[0.3in]
\begin{Schunk}
\begin{Soutput}
75% 
160 
\end{Soutput}
\end{Schunk}
    \end{solutionordottedlines}
    \part[5] What is the variance of scores?
    \begin{solutionordottedlines}[0.3in] 
\begin{Schunk}
\begin{Soutput}
[1] 92.42105
\end{Soutput}
\end{Schunk}
    \end{solutionordottedlines}
    \part[2] What is the standard deviation of scores?
    \begin{solutionordottedlines}[0.3in] 
\begin{Schunk}
\begin{Soutput}
[1] 9.613587
\end{Soutput}
\end{Schunk}
    \end{solutionordottedlines}
    \part[3] What is the coefficient of variance for scores?
    \begin{solutionordottedlines}[0.3in] 
\begin{Schunk}
\begin{Soutput}
[1] 0.1354026
\end{Soutput}
\end{Schunk}
    \end{solutionordottedlines}
    \end{parts}

    \question[3] What is a random variable?
    \begin{solutionordottedlines}[0.3in]
      A variable whose outcome is unpredictable
    \end{solutionordottedlines}
    \question[3] What is the sample space of a random variable?
    \begin{solutionordottedlines}[0.3in]
      The set of all possible outcomes
    \end{solutionordottedlines}
    \question[3] You're randomly flipping through the book Neuromancer by William
    Gibson, closing your eyes and pointing at a word on the page, and
    writing it down.  What is the sample space for this variable?
    \begin{solutionordottedlines}[0.3in]
      All of the words in the book.
    \end{solutionordottedlines}

%

% latex table generated in R 2.13.0 by xtable 1.5-6 package
% Mon Sep 26 00:39:12 2011
\begin{table}[ht]
\begin{center}
\begin{tabular}{llrll}
  \hline
names & gender & age & hair & eyes \\ 
  \hline
Kelly & Female & 46 & Black & Black \\ 
  Nicole & Female & 35 & Black & Blue \\ 
  Ophelia & Female & 37 & Black & Brown \\ 
  Cindy & Female & 33 & Black & Brown \\ 
  Vivian & Female & 31 & Blond & Green \\ 
  Quetzal & Female & 28 & Brown & Black \\ 
  Jasmine & Female & 26 & Brown & Blue \\ 
  Rachel & Female & 48 & Brown & Brown \\ 
  Ursula & Female & 31 & Brown & Brown \\ 
  Hermione & Female & 26 & Brown & Brown \\ 
  Ingrid & Female & 32 & Brown & Brown \\ 
  Elise & Female & 40 & Red & Brown \\ 
  Michael & Male & 27 & Blond & Brown \\ 
  Richard & Male & 35 & Brown & Black \\ 
  Ernest & Male & 30 & Brown & Blue \\ 
  Xavier & Male & 33 & Brown & Blue \\ 
  David & Male & 36 & Brown & Brown \\ 
  Phillip & Male & 36 & Brown & Brown \\ 
  George & Male & 20 & Brown & Brown \\ 
  Harry & Male & 30 & Red & Green \\ 
   \hline
\end{tabular}
\caption{Random people at Grand Central Station}
\label{tab:strangers}
\end{center}
\end{table}%

    \question Table~\ref{tab:strangers} shows data collected on every tenth
    person that walked out of Grand Central Station on a Tuesday
    afternoon.  Assume this is a random variable.
    \begin{parts}
      \part[2] Is the following a subset of brown-haired people?
\begin{displaymath} S_1=\left\{ Ursula, Phillip, Michael, Hermione \right\}\end{displaymath}      \part[2] Is the following a subset of brown-eyed women?
\begin{displaymath} S_2=\left\{ Cindy, Ingrid, Rachel \right\}\end{displaymath}
      \part
      \begin{subparts}
        \subpart[2] Is the following a subset of blue-eyed people?
\begin{displaymath} S_3=\left\{ Xavier, Ernest, Nicole, Jasmine \right\}\end{displaymath}        \subpart[1] Is $S_3$ a \emph{proper} subset of blue-eyed people?
        \begin{solutionordottedlines}[0.3in]
          No, there are no elements in the set of blue-eyed people that are
          not in $S_3$.
        \end{solutionordottedlines}
      \end{subparts}
      \part[3] $H_{black}$ is the set of all black-haired people.  $E_{green}$ is the set of
      all green-eyed people.  Describe (i.e., list the elements of) the
      union: $S_4 = H_{black} \cup E_{green}$
        \begin{solutionordottedlines}[0.3in]
\begin{displaymath} S_4=\left\{ Kelly, Nicole, Ophelia, Cindy, Vivian, Harry \right\}\end{displaymath}        \end{solutionordottedlines}
      \part[3] $G_m$ is the set of all males.  $H_{brown}$ is the set of
      all brown-haired people.  Describe the intersection: $S_5 = G_m \cap H_{brown}$
        \begin{solutionordottedlines}[0.3in]
\begin{displaymath} S_5=\left\{ Richard, Ernest, Xavier, David, Phillip, George \right\}\end{displaymath}        \end{solutionordottedlines}
      \part[3] $G_m$ is the set of all males.  $E_{brown}$ is the set of
      all brown-eyed people.  Describe the set difference: $S_6 = G_m - E_{brown}$
        \begin{solutionordottedlines}[0.3in]
\begin{displaymath} S_6=\left\{ Richard, Ernest, Xavier, Harry \right\}\end{displaymath}        \end{solutionordottedlines}
      \part[3] $G_m$ is the set of all males.  $E_{brown}$ is the set of
      all brown-eyed people.  Describe the set difference: $S_7 = E_{brown}
      - G_m$
        \begin{solutionordottedlines}[0.3in]
\begin{displaymath} S_7=\left\{ Ophelia, Cindy, Rachel, Ursula, Hermione, Ingrid, Elise \right\}\end{displaymath}        \end{solutionordottedlines}
    \end{parts}

    \question[8] Based on Table~\ref{tab:strangers}, create an outcome table
    for the variables of hair color and eye color.
\ifprintanswers
% latex table generated in R 2.13.0 by xtable 1.5-6 package
% Mon Sep 26 00:39:12 2011
\begin{table}[ht]
\begin{center}
\begin{tabular}{|c|c|c|c|c||c|}
  \hline
 & Black & Blue & Brown & Green & Sum \\ 
  \hline
Black & 1 & 1 & 2 & 0 & 4 \\ 
   \hline
Blond & 0 & 0 & 1 & 1 & 2 \\ 
   \hline
Brown & 2 & 3 & 7 & 0 & 12 \\ 
   \hline
Red & 0 & 0 & 1 & 1 & 2 \\ 
   \hline
\hline
Sum & 3 & 4 & 11 & 2 & 20 \\ 
   \hline
\end{tabular}
\caption{Outcome Table for eye and hair color}
\label{tab:outcome1}
\end{center}
\end{table}\else
%
% latex table generated in R 2.13.0 by xtable 1.5-6 package
% Mon Sep 26 00:39:12 2011
\begin{table}[ht]
\begin{center}
\begin{tabular}{|c|c|c|c|c||p{1cm}|}
  \hline
 & Black & Blond & Brown & Red &   \\ 
  \hline
Black &   &   &   &   &   \\ 
   \hline
Blue &   &   &   &   &   \\ 
   \hline
Brown &   &   &   &   &   \\ 
   \hline
Green &   &   &   &   &   \\ 
   \hline
\hline
  &   &   &   &   &   \\ 
   \hline
\end{tabular}
\caption{Outcome Table for eye and hair color}
\label{tab:outcome1}
\end{center}
\end{table}\fi
\newpage
    \question[3] What is the law of large numbers?
    \begin{solutionordottedlines}[0.6in]
      As the number of samples increases, the relative frequency of
      outcomes approaches the true probability of the outcome.
    \end{solutionordottedlines}

% latex table generated in R 2.13.0 by xtable 1.5-6 package
% Mon Sep 26 00:39:12 2011
\begin{table}[ht]
\begin{center}
\begin{tabular}{|c|c|c|c|c||c|}
  \hline
 & Black & Blue & Brown & Green & Sum \\ 
  \hline
Black & 178 & 45 & 143 & 42 & 408 \\ 
   \hline
Blond & 10 & 120 & 161 & 94 & 385 \\ 
   \hline
Brown & 263 & 132 & 495 & 119 & 1009 \\ 
   \hline
Red & 13 & 7 & 58 & 120 & 198 \\ 
   \hline
\hline
Sum & 464 & 304 & 857 & 375 & 2000 \\ 
   \hline
\end{tabular}
\caption{Frequencies of recorded eye and hair color}
\label{tab:outcome2}
\end{center}
\end{table}
    \question Suppose Table~\ref{tab:outcome2} represents the outcome table
    for people's hair and eye color after collecting a lot more data.
    \begin{parts}
      \part[2] Using the law of large numbers, what is the expected probability of
      encountering a person with blue eyes?
      \begin{solutionordottedlines}[0.3in]
[1] "Number of blue-eyed people: 304   "[1] "Number of people: 2000   "[1] "Probability: 0.152   "      \end{solutionordottedlines}
      \part[2] What is the joint probability of encountering a person with red
      hair and green eyes?
      \begin{solutionordottedlines}[0.3in]
[1] "Number of red-haired, green--eyed people: 120   "[1] "Number of people: 2000   "[1] "Probability: 0.06   "      \end{solutionordottedlines}
      \part[2] What is the marginal probability of encountering a person with green eyes?
      \begin{solutionordottedlines}[0.3in]
[1] "Number of green--eyed people: 375   "[1] "Number of people: 2000   "[1] "Probability: 0.1875   "      \end{solutionordottedlines}
      \part[3] What is the conditional probability of encountering a person
      with red hair given they have green eyes?
      \begin{solutionordottedlines}[0.3in]
[1] "Number of red-haired, green--eyed people: 120   "[1] "Number of green--eyed: 375   "[1] "Probability: 0.32   "      \end{solutionordottedlines}
      \part[3] What is the conditional probability of encountering a person
      with black hair given they have blue eyes?
      \begin{solutionordottedlines}[0.3in]
[1] "Number of black-haired, blue--eyed people: 45   "[1] "Number of blue--eyed: 304   "[1] "Probability: 0.148026315789474   "      \end{solutionordottedlines}
      \part[3] Given a person has black hair, what is the probability they
      have blue eyes?
      \begin{solutionordottedlines}[0.3in]
[1] "Number of black-haired, blue--eyed people: 45   "[1] "Number of black-haired: 408   "[1] "Probability: 0.110294117647059   "      \end{solutionordottedlines}
      \part[4] Knowing the probability of a person having blue eyes given they
      have black hair, and the marginal probability of blue eyes, do you
      believe eye color and hair color are independent?  Why or why not?
      \begin{solutionordottedlines}[0.3in]
        They are not independent, because the conditional probability does
        not equal the marginal probability, i.e., $p(E_{blue} \mid
        H_{black}) \neq p(E_{blue})$
      \end{solutionordottedlines}
    \end{parts}
    
    \question[4] The probability that a college student illegaly downloads
    songs and movies, $p(s \cap m)=0.06$.  Sixty out of 100 students
    download songs, and 10 out of 100 students download movies.  Are
    downloading songs and movies independent?
    \begin{solutionordottedlines}[0.3in]
      They are independent, because the probability of the intersection
      equals the product of the probabilities.
    \end{solutionordottedlines}

\ifprintanswers
% latex table generated in R 2.13.0 by xtable 1.5-6 package
% Mon Sep 26 00:39:13 2011
\begin{table}[ht]
\begin{center}
\begin{tabular}{|c|c|c|c|c|}
  \hline
Namet & Gender & Age & $Age - \mu$ & ${(Age - \mu)}^2$ \\ 
  \hline
Rachel & Female & 48 & 15 & 225 \\ 
   \hline
Kelly & Female & 46 & 13 & 169 \\ 
   \hline
Elise & Female & 40 & 7 & 49 \\ 
   \hline
Ophelia & Female & 37 & 4 & 16 \\ 
   \hline
David & Male & 36 & 3 & 9 \\ 
   \hline
Phillip & Male & 36 & 3 & 9 \\ 
   \hline
Nicole & Female & 35 & 2 & 4 \\ 
   \hline
Richard & Male & 35 & 2 & 4 \\ 
   \hline
Cindy & Female & 33 & 0 & 0 \\ 
   \hline
Xavier & Male & 33 & 0 & 0 \\ 
   \hline
Ingrid & Female & 32 & -1 & 1 \\ 
   \hline
Vivian & Female & 31 & -2 & 4 \\ 
   \hline
Ursula & Female & 31 & -2 & 4 \\ 
   \hline
Ernest & Male & 30 & -3 & 9 \\ 
   \hline
Harry & Male & 30 & -3 & 9 \\ 
   \hline
Quetzal & Female & 28 & -5 & 25 \\ 
   \hline
Michael & Male & 27 & -6 & 36 \\ 
   \hline
Jasmine & Female & 26 & -7 & 49 \\ 
   \hline
Hermione & Female & 26 & -7 & 49 \\ 
   \hline
George & Male & 20 & -13 & 169 \\ 
   \hline
\hline
 &  & 33 &  & 840 \\ 
   \hline
\end{tabular}
\caption{Random people at Grand Central Station}
\label{tab:ages}
\end{center}
\end{table}\else
%
% latex table generated in R 2.13.0 by xtable 1.5-6 package
% Mon Sep 26 00:39:13 2011
\begin{table}[ht]
\begin{center}
\begin{tabular}{|c|c|c|p{3cm}|p{3cm}|}
  \hline
Name & Gender & Age &  &  \\ 
  \hline
Rachel & Female & 48 &  &  \\ 
   \hline
Kelly & Female & 46 &  &  \\ 
   \hline
Elise & Female & 40 &  &  \\ 
   \hline
Ophelia & Female & 37 &  &  \\ 
   \hline
David & Male & 36 &  &  \\ 
   \hline
Phillip & Male & 36 &  &  \\ 
   \hline
Nicole & Female & 35 &  &  \\ 
   \hline
Richard & Male & 35 &  &  \\ 
   \hline
Cindy & Female & 33 &  &  \\ 
   \hline
Xavier & Male & 33 &  &  \\ 
   \hline
Ingrid & Female & 32 &  &  \\ 
   \hline
Vivian & Female & 31 &  &  \\ 
   \hline
Ursula & Female & 31 &  &  \\ 
   \hline
Ernest & Male & 30 &  &  \\ 
   \hline
Harry & Male & 30 &  &  \\ 
   \hline
Quetzal & Female & 28 &  &  \\ 
   \hline
Michael & Male & 27 &  &  \\ 
   \hline
Jasmine & Female & 26 &  &  \\ 
   \hline
Hermione & Female & 26 &  &  \\ 
   \hline
George & Male & 20 &  &  \\ 
   \hline
\hline
 &  &  &  &  \\ 
   \hline
\end{tabular}
\caption{Random people at Grand Central Station}
\label{tab:ages}
\end{center}
\end{table}\fi

  \question Now you're interested in the ages of the people from Grand
  Central Station.

    \begin{parts}
      \part[4] What is the average age?
        \begin{solutionordottedlines}[0.3in] 
\begin{Schunk}
\begin{Soutput}
[1] 33
\end{Soutput}
\end{Schunk}
        \end{solutionordottedlines}
    \part[1] What is the range of ages?
    \begin{solutionordottedlines}[0.3in]
\begin{Schunk}
\begin{Soutput}
[1] 28
\end{Soutput}
\end{Schunk}
    \end{solutionordottedlines}
    \part[2] What is the modal age?
    \begin{solutionordottedlines}[0.3in]
\begin{Schunk}
\begin{Soutput}
26 
 2 
\end{Soutput}
\end{Schunk}
    \end{solutionordottedlines}
      \part[2] What is the median age?
        \begin{solutionordottedlines}[0.3in] 
\begin{Schunk}
\begin{Soutput}
[1] 32.5
\end{Soutput}
\end{Schunk}
        \end{solutionordottedlines}
    \part[5] What is the variance of ages?
    \begin{solutionordottedlines}[0.3in] 
\begin{Schunk}
\begin{Soutput}
[1] 44.21053
\end{Soutput}
\end{Schunk}
    \end{solutionordottedlines}
    \part[2] What is the standard deviation of ages?
    \begin{solutionordottedlines}[0.3in] 
\begin{Schunk}
\begin{Soutput}
[1] 6.6491
\end{Soutput}
\end{Schunk}
    \end{solutionordottedlines}
    \part[3] What is the coefficient of variance for ages?
    \begin{solutionordottedlines}[0.3in] 
\begin{Schunk}
\begin{Soutput}
[1] 0.2014879
\end{Soutput}
\end{Schunk}
    \end{solutionordottedlines}
    \end{parts}

    \question[3] What is an \emph{event} in probability?
    \begin{solutionordottedlines}[0.3in] 
      An event is a subset of the sample space.
    \end{solutionordottedlines}

    \question Using the observed frequencies, estimate the following:
    \begin{parts}
      \part[4] What is the probability that a person at Grand Central
      Station is older than 30?
    \begin{solutionordottedlines}[0.3in] 
[1] "Number of people older than 30:  13 /20"[1] "Probability:  0.65   "    \end{solutionordottedlines}
      \part[4] What is the probability that a person at Grand Central
      Station is younger than 35?
    \begin{solutionordottedlines}[0.3in] 
[1] "Number of people younger than 35:  12 /20"[1] "Probability:  0.6   "    \end{solutionordottedlines}
      \part[4] What is the probability that a person at Grand Central
      Station is exactly 35?
    \begin{solutionordottedlines}[0.3in] 
[1] "Number of people exactly 35:  2 /20"[1] "Probability:  0.1   "    \end{solutionordottedlines}
      \part[4] What is the probability that a person at Grand Central
      Station is a male younger than 40?
    \begin{solutionordottedlines}[0.3in] 
[1] "Number of males younger than 40:  8 /20"[1] "Probability:  0.4   "    \end{solutionordottedlines}
      \part[5] What is the probability a person at Grand Central is between
      the age of 18-28 (inclusive)?
    \begin{solutionordottedlines}[0.3in] 
[1] "Number of people between 18-28:  5 /20"[1] "Probability:  0.25"    \end{solutionordottedlines}
      \part[5] Given a person at Grand Central Station is female, what is
      the probability they are between 25-35 (inclusive)?
    \begin{solutionordottedlines}[0.3in] 
[1] "Number of women between 25-35:  8 / 12"[1] "Probability:  0.666666666666667"    \end{solutionordottedlines}
    \end{parts}

    \bonusquestion[5] The probability that it will rain tomorrow, $p(r)=0.3$.
    The probability that you will wash your car tomorrow, $p(w)=0.6$.  The
    probability you will wash your car given it rains tomorrow,
    $p(w \mid r)=0.02$.  What is the probability it will rain tomorrow if
    you wash your car?
    \begin{solutionordottedlines}[0.3in]
      $p(r \mid w)=\frac{p(w \mid r) \cdot \p(r)}{p(w)}=0.01$
    \end{solutionordottedlines}


\end{questions}

\ifprintanswers
\begin{center} \gradetable[v][questions] \end{center}
\fi

\end{document}
