\documentclass[11pt]{exam}
\usepackage{Sweave}
\addtolength{\textwidth}{-0.5in} 
 
\noprintanswers
\addpoints
\pointformat{(\thepoints)}

\begin{document}


\firstpageheader{Name:\enspace\makebox[2in]{\hrulefill} \\ Statistics
  \\ BT221}{\Large Homework 6 } {\ifprintanswers \textbf{Answer Key} \else
  Due 11/3 \fi
  \\ UNIQUEKEY}
\runningheader{Statistics \\ BT221}{\Large Homework 6 }{\ifprintanswers
  \textbf{Answer Key} \fi \\ UNIQUEKEY} 
\headrule 
\cfoot{\thepage}

\begin{questions}

%%%%%%%%  Problem 1

%

  \question You're the manager of an electronics store and want to know if
  any particular configuration of HDTV is espeically popular.  The TVs come
  in 5 different sizes and 3 different kinds of TV: LED, LCD, and
  Plasma.  Table~\ref{tab:hdtv} shows the number of each kind of TV sold.

  \begin{parts}
    \part[2] What is the null hypothesis?
    \begin{solution}
      No particular TV sold more than any other TV
    \end{solution}
    \part[2] What is the alternative hypothesis?
    \begin{solution}
      At least one TV sold more than any of the other TVs
    \end{solution}
      \part[10] What is the chi-squared statistic for these figures?
      \begin{solution}
	Pearson's Chi-squared test

data:  HDTVs 
X-squared = 17.8347, df = 8, p-value = 0.0225      \end{solution}
    \part[2] How many degrees of freedom are there?
    \begin{solution}
      8
    \end{solution}
    \part[5] What is the critical chi-squared value for Type I error rate = 0.05?
    \begin{solution}
[1] 2.732637
    \end{solution}
    \part[5] Can you reject the null hypothesis with a Type I error rate = 0.05?
    \begin{solution}
      maybe?
    \end{solution}
  \end{parts}

%
% latex table generated in R 2.13.0 by xtable 1.5-6 package
% Tue Nov  1 02:03:21 2011
\begin{table}[ht]
\begin{center}
\begin{tabular}{rrrrrr}
  \hline
 & 27" & 31" & 36" & 42" & 52" \\ 
  \hline
LCD &  44 &  99 & 114 & 102 &  97 \\ 
  LED &  66 & 155 & 130 & 104 & 115 \\ 
  Plasma &  44 &  97 & 101 & 102 & 130 \\ 
   \hline
\end{tabular}
\caption{Sales of HDTVs}
\label{tab:hdtv}
\end{center}
\end{table}%    

%%%%%%%%  Problem 2

%

  \question In one year, you randomly select 25 days and measure the
  national sales of lingerie and the number of visitors to online dating
  sites, listed in Table \ref{tab:lingerie}.
    \begin{parts}
      \part[10] What is the covariance between lingerie sales and online
      dating visitors?
      \begin{solution}
[1] 58.65      \end{solution}
      \part[10] What is the correlation coefficient between these variables?
      \begin{solution}
[1] 0.7018538      \end{solution}
    \part[10] Calculate the t-value for the correlation and determine if
    the variables are significantly correlated.
      \begin{solution}
\begin{Schunk}
\begin{Sinput}
> lv.lm <- lm(Lingerie ~ Visitors)
> print(summary(lv.lm))
\end{Sinput}
\begin{Soutput}
Call:
lm(formula = Lingerie ~ Visitors)

Residuals:
    Min      1Q  Median      3Q     Max 
-35.408 -25.751  -0.579  24.592  34.249 

Coefficients:
            Estimate Std. Error t value Pr(>|t|)    
(Intercept)   49.893     22.424   2.225   0.0362 *  
Visitors       8.391      1.776   4.725 9.23e-05 ***
---
Signif. codes:  0 ‘***’ 0.001 ‘**’ 0.01 ‘*’ 0.05 ‘.’ 0.1 ‘ ’ 1 

Residual standard error: 23 on 23 degrees of freedom
Multiple R-squared: 0.4926,	Adjusted R-squared: 0.4705 
F-statistic: 22.33 on 1 and 23 DF,  p-value: 9.23e-05 
\end{Soutput}
\end{Schunk}
      \end{solution}
      \part[12] What would be the expected lingerie sales on a day with
      twenty thousand visitors to online dating sites?
      \begin{solution}
Visitors 
217.7039       \end{solution}
    \end{parts}    


%%%%%%%%  Problem 3

%

  \question For a recent assignment, you took a poll of the class and asked
  how many hours your classmates procrastinated before starting the
  assignment and their grade on the assignment.  The results of your
  poll are shown in Table \ref{tab:procrast}.
    \begin{parts}
      \part[10] What is the correlation coefficient between these variables?
      \begin{solution}
[1] -0.7573811      \end{solution}
    \part[10] Calculate the t-value for the correlation and determine if
    the variables are significantly correlated.
      \begin{solution}
\begin{Schunk}
\begin{Sinput}
> gp.lm <- lm(Grade ~ Procrastination)
> print(summary(gp.lm))
\end{Sinput}
\begin{Soutput}
Call:
lm(formula = Grade ~ Procrastination)

Residuals:
     Min       1Q   Median       3Q      Max 
-10.7387  -3.9616  -0.3387   4.0841   7.6613 

Coefficients:
                Estimate Std. Error t value Pr(>|t|)    
(Intercept)     108.2605     6.0093  18.015 4.66e-15 ***
Procrastination  -4.0152     0.7218  -5.563 1.17e-05 ***
---
Signif. codes:  0 ‘***’ 0.001 ‘**’ 0.01 ‘*’ 0.05 ‘.’ 0.1 ‘ ’ 1 

Residual standard error: 5.588 on 23 degrees of freedom
Multiple R-squared: 0.5736,	Adjusted R-squared: 0.5551 
F-statistic: 30.94 on 1 and 23 DF,  p-value: 1.168e-05 
\end{Soutput}
\end{Schunk}
      \end{solution}
      \part[12] What would be the expected grade for someone who only
      procrastinated 2 hours?
      \begin{solution}
Procrastination 
         100.23       \end{solution}
    \end{parts}


\newpage
.
\newpage

%
% latex table generated in R 2.13.0 by xtable 1.5-6 package
% Tue Nov  1 02:03:21 2011
\begin{table}[ht]
\begin{center}
\begin{tabular}{rrr}
  \hline
 & Lingerie & Visitors \\ 
  \hline
1 & 210 & 18 \\ 
  2 & 210 & 15 \\ 
  3 & 210 & 15 \\ 
  4 & 180 & 12 \\ 
  5 & 180 & 12 \\ 
  6 & 180 & 15 \\ 
  7 & 180 & 15 \\ 
  8 & 150 & 12 \\ 
  9 & 150 & 12 \\ 
  10 & 150 & 12 \\ 
  11 & 150 & 12 \\ 
  12 & 150 & 12 \\ 
  13 & 150 & 15 \\ 
  14 & 150 & 9 \\ 
  15 & 150 & 15 \\ 
  16 & 150 & 9 \\ 
  17 & 150 & 15 \\ 
  18 & 150 & 15 \\ 
  19 & 150 & 9 \\ 
  20 & 150 & 12 \\ 
  21 & 120 & 12 \\ 
  22 & 120 & 9 \\ 
  23 & 120 & 9 \\ 
  24 & 90 & 9 \\ 
  25 & 90 & 9 \\ 
   \hline
\end{tabular}
\caption{Sales of lingerie and vistors to online dating sites (in 1000s)}
\label{tab:lingerie}
\end{center}
\end{table}
%
% latex table generated in R 2.13.0 by xtable 1.5-6 package
% Tue Nov  1 02:03:21 2011
\begin{table}[ht]
\begin{center}
\begin{tabular}{rrr}
  \hline
 & Procrastination & Grade \\ 
  \hline
1 & 11.00 & 55.00 \\ 
  2 & 11.00 & 65.40 \\ 
  3 & 11.00 & 63.00 \\ 
  4 & 9.50 & 74.20 \\ 
  5 & 9.50 & 71.00 \\ 
  6 & 9.50 & 67.80 \\ 
  7 & 9.50 & 74.20 \\ 
  8 & 8.00 & 75.80 \\ 
  9 & 8.00 & 83.80 \\ 
  10 & 8.00 & 79.80 \\ 
  11 & 8.00 & 83.00 \\ 
  12 & 8.00 & 82.20 \\ 
  13 & 8.00 & 70.20 \\ 
  14 & 8.00 & 65.40 \\ 
  15 & 8.00 & 76.60 \\ 
  16 & 8.00 & 82.20 \\ 
  17 & 8.00 & 75.80 \\ 
  18 & 8.00 & 71.80 \\ 
  19 & 8.00 & 70.20 \\ 
  20 & 8.00 & 83.80 \\ 
  21 & 6.50 & 78.20 \\ 
  22 & 6.50 & 88.60 \\ 
  23 & 6.50 & 73.40 \\ 
  24 & 5.00 & 87.00 \\ 
  25 & 5.00 & 87.00 \\ 
   \hline
\end{tabular}
\caption{Hours spent procrastinating before starting an assignment and the final grade}
\label{tab:procrast}
\end{center}
\end{table}%    

\end{questions}

\ifprintanswers
\begin{center} \gradetable[v][questions] \end{center}
\fi

\end{document}
