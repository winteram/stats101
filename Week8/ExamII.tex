\documentclass[11pt]{exam}
\usepackage{Sweave}
\addtolength{\textwidth}{-0.5in} 
 
\noprintanswers
\addpoints
\pointformat{(\thepoints)}

\begin{document}


\firstpageheader{Name:\enspace\makebox[2in]{\hrulefill} \\ Statistics
  \\ BT221}{\Large Exam II } {\ifprintanswers \textbf{Answer Key} \fi
  \\ UNIQUEKEY}
\runningheader{Statistics \\ BT221}{\Large Exam II }{\ifprintanswers
  \textbf{Answer Key} \fi \\ UNIQUEKEY} 
\headrule 
\cfoot{\thepage}

\begin{questions}

%%%%%%%%  Problem 1

%

  \question On the game show, ``The Price is Right'', each contestant spins
  a wheel at the end of their turn, trying to get as close to a dollar
  \emph{without going over}.  Table~\ref{tab:pir} shows how many
  times each value appears on one of the 50 segments on the wheel.
  \begin{parts}
    \part[10] What is the expected value of a single spin?
    \begin{solutionordottedlines}[0.3in]
\begin{Schunk}
\begin{Soutput}
[1] 0.539
\end{Soutput}
\end{Schunk}
    \end{solutionordottedlines}
    \bonuspart[4] Suppose you got \$0.50 on your first spin.  Given that
    you would lose if the sum of your spins is greater than \$1.00, should
    you spin again? Why or why not?
    \begin{solutionordottedlines}[0.3in]
\begin{Schunk}
\begin{Soutput}
[1] "no"
\end{Soutput}
\end{Schunk}
    \end{solutionordottedlines}
  \end{parts}

% latex table generated in R 2.13.0 by xtable 1.5-6 package
% Tue Nov  8 01:22:38 2011
\begin{table}[ht]
\begin{center}
\begin{tabular}{rr}
  \hline
Value & Frequency \\ 
  \hline
0.01 & 1.00 \\ 
  0.10 & 3.00 \\ 
  0.15 & 3.00 \\ 
  0.25 & 4.00 \\ 
  0.35 & 5.00 \\ 
  0.45 & 8.00 \\ 
  0.50 & 5.00 \\ 
  0.60 & 2.00 \\ 
  0.70 & 5.00 \\ 
  0.80 & 2.00 \\ 
  0.85 & 2.00 \\ 
  0.90 & 5.00 \\ 
  0.95 & 3.00 \\ 
  0.99 & 1.00 \\ 
  1.00 & 1.00 \\ 
   \hline
\end{tabular}
\caption{Value and number of times value appears on wheel}
\label{tab:pir}
\end{center}
\end{table}%


%%%%%%%%  Problem 2


\question James Bond is held captive by the evil Dr.~No.  Dr.~No offers
Mr.~Bond a chance to escape, if the fates deem him worthy by beating the
odds of a coin toss.  Unbeknownst to Dr.~No, James has a trick coin that's
biased to turn up heads 
830 times out of 1000 tosses.
\begin{parts}
\part[5] For the first challenge, Mr.~Bond must get exactly 
4 heads in 
6 tosses.  What is the probability James Bond will make it past the first
challenge?
\begin{solutionordottedlines}[0.3in]
\begin{Schunk}
\begin{Soutput}
0.2712775
\end{Soutput}
\end{Schunk}
\end{solutionordottedlines}
\part[5] For the second challenge, Mr.~Bond must turn up at least one
\emph{tail} in 
5 tosses. What is the probability James Bond will make it past the second
challenge?
\begin{solutionordottedlines}[0.3in]
\begin{Schunk}
\begin{Soutput}
0.6060959
\end{Soutput}
\end{Schunk}
\end{solutionordottedlines}
\newpage
\part[5] For the final challenge, Mr.~Bond must turn up his first \emph{tail} on
toss
9 and no earlier. What is the probability James Bond will make it past the final
challenge?
\begin{solutionordottedlines}[0.3in]
\begin{Schunk}
\begin{Soutput}
0.03828897
\end{Soutput}
\end{Schunk}
\end{solutionordottedlines} 
\bonuspart[5] What is the probability James will pass all
three challenges?  
\begin{solutionordottedlines}[0.3in]
\begin{Schunk}
\begin{Soutput}
0.006295479
\end{Soutput}
\end{Schunk}
\end{solutionordottedlines}
\end{parts}


%%%%%%%%  Problem 3

\question[3] You're planning a study on the differences between men and
women's preferences for cats versus dogs as pets.  You go door-to-door and
you want to save time and money, so if a man answers, you try to also ask
his wife, or if a woman answers you also ask her husband.  Is this a random sample?
Why or why not?
\begin{solutionordottedlines}[0.5in]
No, because husbands and wives are not independent samples
\end{solutionordottedlines}


%%%%%%%%  Problem 4

\question[3] You've just taken a poll of attitudes towards Herman Cain,
asking 400 people to rate their likelihood of voting for him on a scale
from 1 to 10.  Pretend this is a continuous variable.  The average rating
in your sample is
5.5 and the sample standard deviation is 
2.  What is the margin of error? \emph{hint: this is the average +/- the
  standard error}
\begin{solutionordottedlines}[0.3in]
5.5  +/-  0.1\end{solutionordottedlines}


%%%%%%%%  Problem 5


  \question You want to know if Mt.~Everest is significantly taller than
  the tallest mountains in U.S.  You take a random sample of 30 of the 100
  tallest mountains in the U.S. and find their height above sea level.  The
  average is
8340m with a standard deviation of
3371m. Mt.~Everest is 8,848m above sea level.

  \begin{parts}
      \part[1] What is the null hypothesis?
      \begin{solutionordottedlines}[0.3in]
        Mt.~Everest is not significantly taller than mountains in the U.S.
      \end{solutionordottedlines}
      \part[1] What is the alternative hypothesis?
      \begin{solutionordottedlines}[0.3in]
        Mt.~Everest is significantly taller than mountains in the U.S.
      \end{solutionordottedlines}
      \part[1] Given the alternative hypothesis, should you do a one-tailed
      or two-tailed test?
      \begin{solutionordottedlines}[0.3in]
        One-tailed test
      \end{solutionordottedlines}
      \part[2] For a Type I error rate of 0.05, what is the critical t-value?
      \begin{solutionordottedlines}[0.3in]
1.710882      \end{solutionordottedlines}
      \part[5] For a Type I error rate of 0.05, what is the confidence interval?
      \begin{solutionordottedlines}[0.3in]
( 7276.38 , 9403.62 )      \end{solutionordottedlines}
      \part[10] What is the t-value for Mt.~Everest relative to the sample
      of U.S. mountains?
      \begin{solutionordottedlines}[0.3in]
[1] 4.370217      \end{solutionordottedlines}
      \part[2] Would you reject the null hypothesis with a Type I error
      rate ($\alpha$) = 0.05?
      \begin{solutionordottedlines}[0.3in]
[1] Yes      \end{solutionordottedlines}
  \end{parts}



%%%%%%%%  Problem 6

%
% latex table generated in R 2.13.0 by xtable 1.5-6 package
% Tue Nov  8 01:22:39 2011
\begin{table}[ht]
\begin{center}
\begin{tabular}{rrr}
  \hline
 & Period.1 & Period.2 \\ 
  \hline
1 & 61.00 & 60.00 \\ 
  2 & 57.00 & 55.00 \\ 
  3 & 63.00 & 50.00 \\ 
  4 & 43.00 & 62.00 \\ 
  5 & 58.00 & 43.00 \\ 
  6 & 54.00 & 71.00 \\ 
  7 & 61.00 & 59.00 \\ 
  8 & 37.00 & 64.00 \\ 
   \hline
\end{tabular}
\caption{Visits to website by
each member of the sample, in the first two-week period and the second (ad
campaign) two-week period.}
\label{tab:adcmpgn}
\end{center}
\end{table}
\question Over a two-week period, you measure how often a selected small
sample of test users visit your website.  You then start an online ad
campaign, and measure how often those same users visit your website.  The
number of visits by the users in the first and second 2-week period are
shown in Table~\ref{tab:adcmpgn}. You want to know if the ad campaign had any
effect on your sample.

  \begin{parts}
      \part[1] What is the null hypothesis?
      \begin{solutionordottedlines}[0.3in]
        There is no significant change before and after the ad campaign.
      \end{solutionordottedlines}
      \part[1] What is the alternative hypothesis?
      \begin{solutionordottedlines}[0.3in]
        There was a significant change after the ad campaign.
      \end{solutionordottedlines}
      \part[1] Given the alternative hypothesis, should you do a one-tailed
      or two-tailed test?
      \begin{solutionordottedlines}[0.3in]
        Two-tailed test
      \end{solutionordottedlines}
      \part[2] For a Type I error rate of 0.05, what is the critical t-value?
      \begin{solutionordottedlines}[0.3in]
one-tailed:  -1.89457860509001 ; two-tailed:  -2.36462425159278      \end{solutionordottedlines}
      \part[10] What is the value of the t-test?
      \begin{solutionordottedlines}[0.3in]
[1] 0.6868028      \end{solutionordottedlines}
      \part[2] Would you reject the null hypothesis with a Type I error
      rate ($\alpha$) = 0.05?
      \begin{solutionordottedlines}[0.3in]
0.514308[1] No      \end{solutionordottedlines}
  \end{parts}

\newpage
%%%%%%%%  Problem 7


\question For your next ad campaign, you recruit visitors to your site and
randomly assign them to see one ad or another ad.  The 44 visitors who saw
the first ad came to your site
63.23 times on average over the next 2 weeks,
with a standard deviation of 
7.26 visits. The 48 visitors who saw the second
ad came to your site 
58.3 times on average over the next 2 weeks, with a
standard deviation of 
7.71 visits.  You want to know which ad is better, if
either.  Assume equal variance in the population.

  \begin{parts}
      \part[1] What is the null hypothesis?
      \begin{solutionordottedlines}[0.3in]
        There is no difference in the number of visits for those who saw Ad
        1 and Ad 2.
      \end{solutionordottedlines}
      \part[1] What is the alternative hypothesis?
      \begin{solutionordottedlines}[0.3in]
        There is a difference in the number of visits for those who saw Ad
        1 and Ad 2.
      \end{solutionordottedlines}
      \part[1] Given the alternative hypothesis, should you do a one-tailed
      or two-tailed test?
      \begin{solutionordottedlines}[0.3in]
        Two-tailed test
      \end{solutionordottedlines}
      \part[2] For a Type I error rate of 0.05, what is the critical t-value?
      \begin{solutionordottedlines}[0.3in]
1.985251      \end{solutionordottedlines}
      \part[10] What is the value of the t-test?
      \begin{solutionordottedlines}[0.3in]
       t 
3.149369       \end{solutionordottedlines}
      \part[2] Would you reject the null hypothesis with a Type I error
      rate ($\alpha$) = 0.05?
      \begin{solutionordottedlines}[0.3in]
0.002220847[1] Yes      \end{solutionordottedlines}
      \part[2] Would you reject the null hypothesis with a Type I error
      rate ($\alpha$) = 0.01?
      \begin{solutionordottedlines}[0.3in]
0.002220847[1] Yes      \end{solutionordottedlines}
  \end{parts}



%%%%%%%%  Problem 8


\question A geologist is given the task of exploring for more oil.  She is
fairly confident one area of tundra has oil buried beneath it, but is
wondering whether the icy plain a few hundred kilometers away might also
contain oil.  She takes 46 measurements of the oil deposits in each
location.  In the tundra, she found an average of 
206.83 liters of oil, with a standard deviation of 
61.04 liters. In the icy plain she found an average of
242.43 liters, with a
standard deviation of 
107.23 liters.  Because the types of earth are so different in the different
 locations, she cannot assume the variability will be the same.

  \begin{parts}
      \part[1] What is the null hypothesis?
      \begin{solutionordottedlines}[0.3in]
        There is no difference in the oil deposits between the tundra and
        the icy plain
      \end{solutionordottedlines}
      \part[1] What is the alternative hypothesis?
      \begin{solutionordottedlines}[0.3in]
        There is a difference in the number of oil deposits between the
        tundra and the icy plain.
      \end{solutionordottedlines}
      \part[1] Given the alternative hypothesis, should you do a one-tailed
      or two-tailed test?
      \begin{solutionordottedlines}[0.3in]
        Two-tailed test
      \end{solutionordottedlines}
      \part[2] For a Type I error rate of 0.05, what is the critical t-value?
      \begin{solutionordottedlines}[0.3in]
-1.98729      \end{solutionordottedlines}
      \part[10] What is the value of the t-test?
      \begin{solutionordottedlines}[0.3in]
        t 
-1.935515       \end{solutionordottedlines}
      \part[2] Would you reject the null hypothesis with a Type I error
      rate ($\alpha$) = 0.05?
      \begin{solutionordottedlines}[0.3in]
0.05697862[1] No      \end{solutionordottedlines}
      \part[2] Would you reject the null hypothesis with a Type I error
      rate ($\alpha$) = 0.01?
      \begin{solutionordottedlines}[0.3in]
0.05697862[1] No      \end{solutionordottedlines}
  \end{parts}


%%%%%%%%  Problem 9


%

  \question The orientation program at your university is meant to attract
  students from every college in the school.  The organizers took a survey
  of 100 students that attended the oritentation, and Table~\ref{tab:college}
  shows how many came from each college. 
  \begin{parts}
    \part[1] What is the null hypothesis?
    \begin{solutionordottedlines}[0.3in]
      All colleges attended equally
    \end{solutionordottedlines}
    \part[1] What is the alternative hypothesis?
    \begin{solutionordottedlines}[0.3in]
      At least one college attended more than the others
    \end{solutionordottedlines}
    \part[10] What is the chi-squared statistic for these figures?
    \begin{solutionordottedlines}[0.3in]
	Chi-squared test for given probabilities

data:  Colleges 
X-squared = 6.08, df = 5, p-value = 0.2985    \end{solutionordottedlines}
    \part[2] How many degrees of freedom are there?
    \begin{solutionordottedlines}[0.3in]
[1] 5    \end{solutionordottedlines}
    \part[2] What is the critical chi-squared value for Type I error rate = 0.05?
    \begin{solutionordottedlines}[0.3in]
11.0705    \end{solutionordottedlines}
    \part[2] Can you reject the null hypothesis with a Type I error rate = 0.05?
    \begin{solutionordottedlines}[0.3in]
0.2985122[1] No    \end{solutionordottedlines}
  \end{parts}


% latex table generated in R 2.13.0 by xtable 1.5-6 package
% Tue Nov  8 01:22:39 2011
\begin{table}[ht]
\begin{center}
\begin{tabular}{rr}
  \hline
 & College.df \\ 
  \hline
Arts &  23 \\ 
  Engineering &  11 \\ 
  Law &  21 \\ 
  Medicine &  16 \\ 
  Sciences &  15 \\ 
  Social Work &  14 \\ 
   \hline
\end{tabular}
\caption{Orientation Attendance by College}
\label{tab:college}
\end{center}
\end{table}
   

%%%%%%%%  Problem 10

%

  \question You're interested in how well the prices of a commodity in a
  futures market predict the prices of the commodity after a year.  The
  prices of the commodity in the futures market, and of the actual
  commodity are shown in Table \ref{tab:futures}.
    \begin{parts}
      \part[5] What is the correlation coefficient between the futures
      prices and the commodity prices?
      \begin{solutionordottedlines}[0.3in]
[1] -0.7071068      \end{solutionordottedlines}
    \part[5] Calculate the t-value for the correlation
      \begin{solutionordottedlines}[0.3in]
\begin{Schunk}
\begin{Soutput}
[1] -2.236068
\end{Soutput}
\end{Schunk}
      \end{solutionordottedlines}
    \part[4] Are the variables siginificantly correlated at the 5\% Type I
    error rate?
      \begin{solutionordottedlines}[0.3in]
\begin{Schunk}
\begin{Soutput}
[1] 0.09490167
\end{Soutput}
\end{Schunk}
      \end{solutionordottedlines}
      \bonuspart[5] What would be the expected commodity price a year later if
      its current futures price is 
12?
      \begin{solutionordottedlines}[0.3in]
 Futures 
6.333333       \end{solutionordottedlines}
    \end{parts}

%
% latex table generated in R 2.13.0 by xtable 1.5-6 package
% Tue Nov  8 01:22:39 2011
\begin{table}[ht]
\begin{center}
\begin{tabular}{rlrr}
  \hline
 & Commodity.Name & Futures & Commodity \\ 
  \hline
1 & Oil & 11.00 & 10.00 \\ 
  2 & Gold & 10.00 & 5.00 \\ 
  3 & Corn & 6.00 & 20.00 \\ 
  4 & Sorgham & 6.00 & 15.00 \\ 
  5 & Soy & 6.00 & 14.00 \\ 
  6 & Steel & 5.00 & 14.00 \\ 
  7 & Pork Bellies & 5.00 & 13.00 \\ 
   \hline
\end{tabular}
\caption{Prices of a commodity in the futures market and its price a year later}
\label{tab:futures}
\end{center}
\end{table}%    

\end{questions}

\begin{center} \gradetable[v][questions] \end{center}

\end{document}
